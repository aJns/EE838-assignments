\setcounter{secnumdepth}{0}

\section{Answers to presented questions}

\subsection{A. Briefly explain about ‘homography (plane projective
transformation)’, and ‘homography matrix’.  What is the DoF (degree of freedom)
of a homography matrix?}

Homography maps points in the plane of the real world to points on the image
plane. The homography matrix describes this mapping, and it can be used to
translate world coordinates to image coordinates. This can be done by simply
multiplying the image coordinates with the matrix.

% explain this better maybe
The degrees of freedom of a homography matrix is given by the count of elements
that are unknown, or free to change. The homography matrix has the form
\(3\times3\), and one of those elements is the scale constant 1, so we have
\(3\times3-1=8\) degrees of freedom.

\subsection{B. What is the difference between ‘region detectors’ and ‘region
descriptors’?}

A region detector detects simple features in an image. These simple features
are often edges or corners, places where the intensity gradient is large.
A region descriptor takes these detected features, and tries to group them. The
purpose of region descriptors is to make the feature detection more robust to
scale, orientation and intensity changes.

\subsection{C. What is the difference between ‘Harris-Laplace’ and
‘Harris-affine’ region detectors? And, what is the difference between ‘Harris-’
and ‘Hessian-’ region detectors?}
%   i. You may find some hints in the following paper - [2] Tuytelaars et al.,
%   “Local invariant feature detectors: A survey”, Computer graphics and
%   vision, 2008.
`Harris-Laplace' is a scale invariant corner detector, `Harris-affine' is
an affine invariant detector.
`Harris-' detectors detect corners, `Hessian-' detectors detect blobs.

\subsection{D. Write the definition of ‘repeatability rate’, which is defined in
the paper [1]. Briefly explain about the method of measuring a repeatability
rate from a pair of images.}

The `repeatability rate´ is the percentage of detected points simultaneously
present in two images. The repeatability rate is of course the better the
higher it is, since more correctly detected shared features makes for better
matching.

The repeatability rate can be calculated by dividing the number of points
repeated in both images by the total number of detected points.

\subsection{E. What is the ground truth for measuring ‘recall’ and ‘1-precision’
in [1]?}

% Recall is computed with respect to the number of corresponding regions and
% 1-precision with respect to the total number of matches.

The ground truth is gotten by manually selecting corresponding regions from
different images of the same scene.

\subsection{F. Write the mathematical expressions of calculating ‘recall’, and
‘1-precision’ in [1], along with brief explanations.}

The mathematical expression for calculating recall is shown in~\ref{eq:recall}.
\#correspondences is the number of manually selected points that are
present in both images. \#correct matches is the number of correspondenses made
by the descriptors that match the manually selected correspondences.
\begin{equation}
  recall = \frac{\#correct\ matches}{\#correspondences}
  \label{eq:recall}
\end{equation}

The mathematical expression for calculating 1-precision is shown
in~\ref{eq:1-precision}. 
\begin{equation}
  1-precision = \frac{\#false\ matches}{\#correct\ matches + \#false\ matches}
  \label{eq:1-precision}
\end{equation}

\subsection{G. There are three typical matching strategies for descriptor
matching (4.1.1 in [1]). Explain each matching strategy, i.e., threshold-based,
nearest neighbor-based, and distance ratio of nearest neighbors.}

\subsection{H. Briefly explain about the dataset for evaluation, which is
provided in [1] with six types.}
