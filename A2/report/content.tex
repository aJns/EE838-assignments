\setcounter{secnumdepth}{0}

\section{Answers to presented questions}

\subsection{A. Briefly explain about ‘homography (plane projective
transformation)’, and ‘homography matrix’.  What is the DoF (degree of freedom)
of a homography matrix?}

Homography maps points in the plane of the real world to points on the image
plane. The homography matrix describes this mapping, and it can be used to
translate world coordinates to image coordinates. This can be done by simply
multiplying the image coordinates with the matrix.

% explain this better maybe
The degrees of freedom of a homography matrix is given by the count of elements
that are unknown, or free to change. The homography matrix has the form
\(3\times3\), and one of those elements is the scale constant 1, so we have
\(3\times3-1=8\) degrees of freedom.

\subsection{B. What is the difference between ‘region detectors’ and ‘region
descriptors’?}

A region detector detects simple features in an image. These simple features
are often edges or corners, places where the intensity gradient is large.
A region descriptor takes these detected features, and tries to group them. The
purpose of region descriptors is to make the feature detection more robust to
scale, orientation and intensity changes.

\subsection{C. What is the difference between ‘Harris-Laplace’ and
‘Harris-affine’ region detectors? And, what is the difference between ‘Harris-’
and ‘Hessian-’ region detectors?}
%   i. You may find some hints in the following paper - [2] Tuytelaars et al.,
%   “Local invariant feature detectors: A survey”, Computer graphics and
%   vision, 2008.
`Harris-Laplace' is a scale invariant corner detector, `Harris-affine' is
an affine invariant detector.
`Harris-' detectors detect corners, `Hessian-' detectors detect blobs.

\subsection{D. Write the definition of ‘repeatability rate’, which is defined in
the paper [1]. Briefly explain about the method of measuring a repeatability
rate from a pair of images.}

The `repeatability rate´ is the percentage of detected points simultaneously
present in two images. The repeatability rate is of course the better the
higher it is, since more correctly detected shared features makes for better
matching.

The repeatability rate can be calculated by dividing the number of points
repeated in both images by the total number of detected points.

\subsection{E. What is the ground truth for measuring ‘recall’ and ‘1-precision’
in [1]?}

% Recall is computed with respect to the number of corresponding regions and
% 1-precision with respect to the total number of matches.

The ground truth is gotten by manually selecting corresponding regions from
different images of the same scene.

\subsection{F. Write the mathematical expressions of calculating ‘recall’, and
‘1-precision’ in [1], along with brief explanations.}

The mathematical expression for calculating recall is shown in~\ref{eq:recall}.
\#correspondences is the number of manually selected points that are
present in both images. \#correct matches is the number of correspondenses made
by the descriptors that match the manually selected correspondences.
\begin{equation}
  recall = \frac{\#correct\ matches}{\#correspondences}
  \label{eq:recall}
\end{equation}

The mathematical expression for calculating 1-precision is shown
in~\ref{eq:1-precision}. The \#correct matches variable is the amount of
detected matches that are the same as the ground truth, and \#false matches are
the matches that aren't the same.
\begin{equation}
  1-precision = \frac{\#false\ matches}{\#correct\ matches + \#false\ matches}
  \label{eq:1-precision}
\end{equation}

\subsection{G. There are three typical matching strategies for descriptor
matching (4.1.1 in [1]). Explain each matching strategy, i.e., threshold-based,
nearest neighbor-based, and distance ratio of nearest neighbors.}

With a \textbf{threshold-based} matching strategy, the regions are matched if
the distance of their descriptors is below a certain threshold. A region can
have several matches.

A \textbf{nearest neighbor-based} matching strategy matches the regions which
are closest to each other, as long as the distance between is also below a set
threshold. Using this strategy a region only has one match.

In a \textbf{distance ratio of nearest neighbors} strategy the closest
descriptors are also matched, but only if the ratio of the distance to the
closest descriptor divided by the distance to the second closest descriptor is
below a threshold.

\subsection{H. Briefly explain about the dataset for evaluation, which is
provided in [1] with six types.}

The dataset contains six types of image pairs:
\begin{enumerate}

  \item Rotated pairs, where the other image is a rotated version of the first
    one.

  \item Rotated and zoomed pairs, where the zoom level also differs.

  \item Pairs where the viewpoint changes, ie.\ the other image is taken from
    another position.

  \item Pairs where the other image is blurred.

  \item JPEG compression on one image

  \item Light change pairs, ie.\ pairs where one image has a lower intensity
    than the other.

\end{enumerate}
The images are also of different types of scenes; Structured scenes with clear
objects and regions, and scenes of repeated textures. There's always a
homography between the images, because they are either taken of the same planar
scene, or the camera position does not vary.



\section{Questions about the code}

\subsection{1. Explain about the data in the file ‘img1.haraff.sift’. What do
the first two lines indicate?  For each line after the 3rd one, what are the
components?}

\subsection{2. Analyze the differences among the three matching strategies
based on the experimental results.}









