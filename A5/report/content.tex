
\section{Algorithm explanation}

\subsection{AKAZE}

AKAZE is an \textbf{Accelerated} version of KAZE\@. It's faster because it builds
it's nonlinear scale space representation using a mathematic framework called
Fast Explicit Diffusion. It has the following steps:
\begin{easylist}[tractatus]

# Building a Nonlinear Scale Space with Fast Explicit Diffusion

## We first need the contrast factor \(\lambda\), which we get from an
algorithm defined in the AKAZE paper. That algorithm uses a gaussian smoothed
input image.

## The next step is to run the FED cycles, which are explicit diffusion schemes
with varying time steps.

### In AKAZE, the FED scheme is embedded in a fine to coarse pyramidal approach
which increases computation speed and results in a set of filtered images.

# Feature Detection

## Compute the determinant of Hessian using the set of filtered images obtained
from the previous step.

# Feature Description

## AKAZE uses a Modified-Local Difference Binary (M-LDB) descriptor. 

### The descriptor uses gradient and intensity information from the nonlinear
scale space.

\end{easylist}

\subsection{LIOP}

\begin{easylist}

\end{easylist}

\subsection{OIOP}

\begin{easylist}

\end{easylist}

\section{Answers to given questions}

\begin{itemize}
  \item \textbf{Explain the difference between SIFT and A-KAZE for detecting local
    maxima.}
  \item \textbf{Explain the difference between LIOP and OIOP descriptors.}
  \item \textbf{As we can see in (2), LIOP descriptor is rotation-invariant by
    itself.  Explain the technical reason of the property.}
\end{itemize}
