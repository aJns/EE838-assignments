\section{Must know}

\subsection{Line equation}

If the normal vector is (a,b) and given a point (c,d), the line equation is
\(a(x-c) + b(y-d)=0\)


\subsection{Matrix properties}

\subsubsection{Homography matrix}
H has 8 degrees of freedom because it's defined up to scale.
2 for scale, 2 for rotation, 2 for translation, 2 for ``line at infinity'' to
finite. Has rank 3.

\subsubsection{Isometry matrix}
Has 3 DoF, 1 rotation, 2 translation

\subsubsection{Similarity matrix}
4 DoF, 1 scale, 1 rotation, 2 translation

\subsubsection{Affine matrix}
6 DoF, 2 scale, 2 rotation, 2 translation

\subsubsection{Fundamental matrix}
Has 7 DoF. 1 DoF is subtracted because it's defined up to scale, 1 because it's
a rank 2 matrix. Because F is singular, it can't be of rank 3. Because it
represents a mapping from a 2d to a 1d projective space, it has to be of rank
2.

\subsubsection{Essential matrix}
The essential matrix has 5 DoF. It's a product of the camera translation and
rotation matrices, which bot have 3 DoF. However, there's scale ambiguity, thus
E has only 5 DoF. E has rank 2, because it is constructed from the translation
matrix, which has rank 2.

\subsubsection{Camera (projection) matrix, and its parts}
A \textit{general} projective camera has rank 3, and 11 DoF.
``
The rank 3 requirement arises because if the rank is less
than this then the range of the matrix mapping will be a
line or point and not the whole plane; in other words not
a 2D image.
''
P has 11 DoF, 5 for K (intrinsic params), 3 for \textbf{t} (translation
vector), 3 for R (rotation matrix).


\subsection{Describing a conic}

The general conic in 3 dimensions is given by
\[ax^2 + bxy + cy^2 + dxz + eyz + fz^2 = 0\]
which can be written using 2D homogeneous coordinates as
\[boldsymbol{x}^T C \boldsymbol{x} = 0\]
where
\[ C = \begin{bmatrix}
        a   &   b/2     &   d/2     \\
        b/2 &   c       &   e/2     \\
        d/2 &   e/2     &   f       
    \end{bmatrix}
\]
5 DoF because up to scale


\subsection{Calculating the camera matrix}

\begin{enumerate}
    \item Get F
    \item Get K, and K' if different cameras were used
    \item Get E from \(E = K^T F K\)
    \item Get R and \textbf{t}
        \begin{enumerate}
            \item Get SVD of E \[ [U, \Sigma, V] = \operatorname{SVD}(E) \]
            \item define W and Z
                \[
                    W = \begin{bmatrix}
                        0   &   -1  & 0 \\
                        1   &   0   & 0 \\
                        0   &   0   & 1
                    \end{bmatrix}
                    Z = \begin{bmatrix}
                        0   &   1   & 0 \\
                        -1  &   0   & 0 \\
                        0   &   0   & 0
                    \end{bmatrix}
                \]
            \item Get \( {[T]}_x = U Z U^T\)
            \item Get \(\boldsymbol{t} = \begin{bmatrix}
                        {[T]}_x(3,2) \\
                        {[T]}_x(1,3) \\
                        {[T]}_x(2,1)
                \end{bmatrix}\)
            \item Get \(R = U W V^T\)
        \end{enumerate}
    \item Get \( P = K [R | \boldsymbol{t}] \)
\end{enumerate}



\subsection{Line \& plane transformations}

\subsubsection{Transformation of lines}

Under a point transformation \( x_i' = H x_i \)
\[ l'^T x_i' = l^T H^{-1} x_i = 0 \]
\[ l' = H^{-T} l \]

\subsubsection{Transformation of conics}

Under a point transformation \( x_i' = H x_i \)
\[ x^T C x = x'^T {[H^{-1}]}^T C H^{-1} x' 
    = x'^T H^{-T} C H^{-1} x'
\]
which is a quadratic form \( x'^T C' x' \) with \(C' = H^{-T} C H^{-1} \)

Under a point transformation \( x' = H x \), a conic C transforms to \( C' =
H^{-T} C H^{-1} \)

\section{Important bits by lecture}
