\section{Data generation}

The test data is generated around a manually chosen ground truth model (line or circle).
I've generated the following data set for testing:
\begin{itemize}
  \item As there's 4 different inlier ratios, and 4 different thresholds, I've
    created 16 different data sets per model.
  \item Each sets holds 200 data points in the range of x = [-50, 50].
  \item The given ratio of data points is inside the given threshold. The rest
    are scattered randomly.
  \item The circle and line models (naturally) use different test data sets, so
    there's 32 data sets in total. The different algorithms use the same data.
\end{itemize}

\section{Testing procedure}

I've implemented the RANSAC algorithm for the line and circle models. I've also
implemented the bonus algorithms R-RANSAC and MSAC, both for line and circle
approximation. Each of these algorithms is run 16 times; Once for each dataset.

\section{Effect of parameters on RANSAC performance}

The parameters we are varying are the ratio of inliers in the dataset, and the
inlier threshold ratio.

\section{Comparison of the RANSAC algorithms}
