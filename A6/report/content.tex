\section{Data generation}

The test data is generated around a manually chosen ground truth model (line or circle).
I've generated the following data set for testing:
\begin{itemize}
  \item 200 data points in the range of x = [-50, 50]
  \item These data points are divided into 4 bins
    \begin{itemize}
      \item \%30 of the points are within a distance of 0.5 of the ground truth
        model
      \item \%50 within a distance of 1
      \item \%70 within a distance of 2
      \item \%90 within a distance of 5
      \item and the rest within a distance of 99
    \end{itemize}
  \item The circle and line models (naturally) use different test data sets
\end{itemize}

\section{Testing procedure}

I've implemented the RANSAC algorithm for the line and circle models. I've also
implemented the bonus algorithms R-RANSAC and MSAC, both for line and circle
approximation. Each of these algorithms is run 4 times; Once for each given
threshold value
\(
  \begin{bmatrix} 0.5 & 1 & 2 & 5 \end{bmatrix}
\)

\section{Effect of parameters on RANSAC performance}

The RANSAC threshold is the parameter which seems to have the greatest impact
on approximation accuracy. 
